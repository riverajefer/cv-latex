% YAAC Another Awesome CV LaTeX Template
%
% This template has been downloaded from:
% https://github.com/darwiin/yaac-another-awesome-cv
%
% Author:
% Christophe Roger
%
% Template license:
% CC BY-SA 4.0 (https://creativecommons.org/licenses/by-sa/4.0/)
%Section: Work Experience at the top
\sectionTitle{Experiencia Laboral}{\faSuitcase}
%\renewcommand{\labelitemi}{$\bullet$}
\begin{experiences}
  \experience
    {2022 - Actual}  {Software developer} {Empresa Imagemaker}{Cliente Sovos}
    {} {
        \begin{itemize}
          \item Desarrollé y mantuve sistemas de identificación biométrica (huella, rostro, enrolamiento), integrando soluciones frontend y backend. 
          \item Implementé funcionalidades en aplicaciones web y de escritorio usando React, Angular, Vue.js y Electron, mejorando usabilidad y rendimiento.
          \item Optimicé APIs y sistemas legacy en Python y PHP (Laravel, Yii), aumentando la eficiencia y experiencia del usuario.
        \end{itemize}
       }
       { Python, Go, Angular, Reactjs, Docker, Pipelines, Azure, Laravel}
  \emptySeparator
  \experience
    {}   {Desarrollador Fullstack}{Empresa Imagemaker}{Cliente interno}
    {} {
        \begin{itemize}
          \item Desarrollé un sistema backoffice para gestión operativa (usuarios, permisos, presupuestos), mejorando procesos internos.
          \item Implementé APIs REST con Node.js y MongoDB, y pipelines CI/CD en Azure DevOps, optimizando rendimiento y entregas.
          \item Creé y administré pipelines de CI/CD en Azure DevOps, automatizando el despliegue continuo y reduciendo los tiempos de entrega de nuevas funcionalidades.
          \item Lideré prácticas de desarrollo: mentoría, code reviews y facilitación de ceremonias ágiles, fortaleciendo la colaboración del equipo.
        \end{itemize}
       }
       { Nodejs, Reactjs, Mongodb, Azure DevOps, Scrum }
  \emptySeparator  
  \experience
    {2015 - 2022} {Desarrollador Fullstack}{Empresa Coink} {Cliente interno}
    {}    {

          \begin{itemize}
              \item Lideré el desarrollo de un ecosistema fintech integral, diseñando e implementando desde dispositivos IoT de ahorro (Raspberry Pi, Arduino) hasta las aplicaciones de usuario final.
              \item Optimicé la experiencia de ahorro digital desarrollando interfaces web (Angular, Vue.js) y una billetera móvil (Ionic), resultando en una mayor interacción y satisfacción del cliente.
              \item Creé la arquitectura de comunicación en tiempo real entre hardware y software mediante APIs en Python y Socket.io, habilitando la funcionalidad clave del sistema de alcancías.
            \end{itemize}
        }
        {
          Python, Angular, Vue.js, Ionic Framework, Raspberry Pi, Arduino, Docker, Gitlab CI/CD
        }
  \emptySeparator
  \experience
    {2014-2015} {Desarrollador Web}{Empresa Sionica} {Cliente interno}
    {} {
        \begin{itemize}
          \item Desarrollé diversos aplicativos y sistemas de información web, diseñando soluciones personalizadas que cumplieron con los requerimientos técnicos y las necesidades del negocio.
          \item Integré APIs para conectar sistemas y mejorar la funcionalidad de las aplicaciones,optimizando los flujos de trabajo y la interacción entre plataformas.
        \end{itemize}
      }
      {PHP, Laravel, Angular, CSS, HTML, Javascript, MySQL, Consumo y creación de APIs}
  \emptySeparator
  \experience
    {2014-2015} {Desarrollador Web}{Empresa Donar Cortes} {Cliente interno}
    {} {
        \begin{itemize}
          \item Desarrollé aplicaciones web, sistemas de información y páginas web corporativas, brindando soluciones digitales adaptadas a las necesidades de los clientes.
          \item Implementé backoffice corporativos utilizando tecnologías como Javascript, PHP, HTML, CSS, MySQL, Joomla, WordPress y Moodle, asegurando una gestión eficiente de contenidos y datos.
        \end{itemize}
      }
      {Javascript, PHP, HTML, CSS, MySQL, Joomla, WordPress, Moodle}
  \emptySeparator

  \experience
  {2013 - 2014}  {Docente de Electrónica}{Instituto Técnico Central}{Bogotá}
  {}   {
          Práctica Docente en Electrónica  
          \begin{itemize}
            \item Impartí clases prácticas y teóricas sobre conceptos básicos de electrónica a estudiantes de bachillerato, fomentando el aprendizaje activo y el pensamiento crítico
            \item Diseñé y supervisé experimentos prácticos que permitieron a los estudiantes aplicar sus conocimientos en electrónica de manera tangible y efectiva.
          \end{itemize}
     }
     {Electrónica, Didáctica, Diseño de Experimentos}
\end{experiences}
